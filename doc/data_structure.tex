%%% Template originaly created by Karol Kozioł (mail@karol-koziol.net) and modified for ShareLaTeX use

\documentclass[a4paper,11pt]{article}

\usepackage[T1]{fontenc}
\usepackage[utf8]{inputenc}
\usepackage{graphicx}
\usepackage{xcolor}
\usepackage{longtable} % for 'longtable' environment
\usepackage{pdflscape} % for 'landscape' environment
\usepackage{booktabs}% http://ctan.org/pkg/booktabs
\newcommand{\tabitem}{~~\llap{\textbullet}~~}
\usepackage{xltabular}
\usepackage{enumitem}
% for line spacing in paragraph of itemize
\usepackage{etoolbox} % <--- added
\AtBeginEnvironment{itemize}{\linespread{.7}\selectfont}% <--- added

\renewcommand\familydefault{\sfdefault}
\usepackage{tgheros}
\usepackage[defaultmono]{droidmono}

\usepackage{amsmath,amssymb,amsthm,textcomp}
\usepackage{enumerate}
\usepackage{multicol}
\usepackage{tikz}

\usepackage{geometry}
\geometry{left=25mm,right=25mm,%
bindingoffset=0mm, top=20mm,bottom=20mm}
\newcommand{\tablistcommand}{%
  \leavevmode\par\vspace{-\baselineskip}%
}


\linespread{1.3}

\newcommand{\linia}{\rule{\linewidth}{0.5pt}}

% custom theorems if needed
\newtheoremstyle{mytheor}
    {1ex}{1ex}{\normalfont}{0pt}{\scshape}{.}{1ex}
    {{\thmname{#1 }}{\thmnumber{#2}}{\thmnote{ (#3)}}}

\theoremstyle{mytheor}
\newtheorem{defi}{Definition}

% my own titles
\makeatletter
\renewcommand{\maketitle}{
\begin{center}
\vspace{2ex}
{\huge \textsc{\@title}}
\vspace{1ex}
\\
\linia\\
\@author \hfill \@date
\vspace{4ex}
\end{center}
}
\makeatother
%%%

% custom footers and headers
\usepackage{fancyhdr}
\pagestyle{fancy}
\lhead{}
\chead{}
\rhead{}
\cfoot{Data Structures and Algorithms Summary}
\lfoot{B. S. Chan}
\rfoot{Page \thepage}
\renewcommand{\headrulewidth}{0pt}
\renewcommand{\footrulewidth}{0pt}
%

% code listing settings
\usepackage{listings}
\lstset{
    language=Python,
    basicstyle=\ttfamily\small,
    aboveskip={1.0\baselineskip},
    belowskip={1.0\baselineskip},
    columns=fixed,
    extendedchars=true,
    breaklines=true,
    tabsize=4,
    prebreak=\raisebox{0ex}[0ex][0ex]{\ensuremath{\hookleftarrow}},
    frame=lines,
    showtabs=false,
    showspaces=false,
    showstringspaces=false,
    keywordstyle=\color[rgb]{0.627,0.126,0.941},
    commentstyle=\color[rgb]{0.133,0.545,0.133},
    stringstyle=\color[rgb]{01,0,0},
    numbers=left,
    numberstyle=\small,
    stepnumber=1,
    numbersep=10pt,
    captionpos=t,
    escapeinside={\%*}{*)}
}

%%%----------%%%----------%%%----------%%%----------%%%

\begin{document}

\title{Data Structure and Algorithms Summary}

\author{B. S. Chan}

\date{\today}

%%%%% Stack %%%%%
\newcommand{\stacknotes}{
\begin{minipage}[t]{\linewidth}
\begin{itemize}[topsep=1pt,partopsep=.5ex,parsep=0ex, leftmargin=3mm]
\item Last-in, first out (LIFO)
\item Underflows if attempt to pop an empty stack
\item Overflows if attempt to push to a full stack
\end{itemize}
\vspace{.05cm}
\end{minipage}
}
\newcommand{\stackoperations}{
\begin{minipage}[t]{\linewidth}
\begin{itemize}[topsep=1pt,partopsep=.5ex,parsep=0ex, leftmargin=3mm]
\item push(), i.e., insert, to add new element to top
\item pop(), i.e., delete, to remove top element
\item is\_empty() to check if stack is empty
\end{itemize}
\vspace{.05cm}
\end{minipage}
}
\newcommand{\stacktime}{
\begin{minipage}[t]{\linewidth}
\begin{itemize}[topsep=1pt,partopsep=.5ex,parsep=0ex, leftmargin=3mm]
\item $\mathcal O(1)$ for push
\item $\mathcal O(1)$ for pop
\end{itemize}
\vspace{.05cm}
\end{minipage}
}
\newcommand{\stackspace}{
\begin{minipage}[t]{\linewidth}
\begin{itemize}[topsep=1pt,partopsep=.5ex,parsep=0ex, leftmargin=3mm]
\item $\mathcal O(1)$ for push
\item $\mathcal O(1)$ for pop
\end{itemize}
\vspace{.05cm}
\end{minipage}
}
%%%%% Queue %%%%%
\newcommand{\queuenotes}{
\begin{minipage}[t]{\linewidth}
\begin{itemize}[topsep=1pt,partopsep=.5ex,parsep=0ex, leftmargin=3mm]
\item First-in, first out (FIFO)
\item Underflows if attempt to dequeue an empty queue
\item Overflows if attempt to enqueue a full queue
\end{itemize}
\vspace{.05cm}
\end{minipage}
}
\newcommand{\queueoperations}{
\begin{minipage}[t]{\linewidth}
\begin{itemize}[topsep=1pt,partopsep=.5ex,parsep=0ex, leftmargin=3mm]
\item enqueue(), i.e., insert, to add new element to back of the queue
\item dequeue(), i.e., delete, to remove from front of the queue
\item is\_empty() to check if stack is empty 
\end{itemize}
\vspace{.05cm}
\end{minipage}
}
\newcommand{\queuetime}{
\begin{minipage}[t]{\linewidth}
\begin{itemize}[topsep=1pt,partopsep=.5ex,parsep=0ex, leftmargin=3mm]
\item $\mathcal O(1)$ for enqueue
\item $\mathcal O(1)$ for dequeue
\end{itemize}
\vspace{.05cm}
\end{minipage}
}
\newcommand{\queuespace}{
\begin{minipage}[t]{\linewidth}
\begin{itemize}[topsep=1pt,partopsep=.5ex,parsep=0ex, leftmargin=3mm]
\item $\mathcal O(n)$
\end{itemize}
\vspace{.05cm}
\end{minipage}
}
%%%%% Binary Search Tree %%%%%
\newcommand{\bstnotes}{
\begin{minipage}[t]{\linewidth}
\begin{itemize}[topsep=1pt,partopsep=.5ex,parsep=0ex, leftmargin=3mm]
\item each node contains key, as well as links to left, right, and parent nodes. 
\item \emph{None} if left, right, or parent node is missing
\item \emph{root} node is the only node that should have parent node missing
\item \emph{binary search tree property} if $y$ is in the left (right) subtree of $x$, then $y.key\leq (\geq) x.key$
\end{itemize}
\vspace{.05cm}
\end{minipage}
}
\newcommand{\bstoperations}{
\begin{minipage}[t]{\linewidth}
\begin{itemize}[topsep=1pt,partopsep=.5ex,parsep=0ex, leftmargin=3mm]
\item search()
\item minimum()
\item maximum()
\item predecessor()
\item successor()
\item insert()
\item delete()
\end{itemize}
\vspace{.05cm}
\end{minipage}
}
\newcommand{\bsttime}{
\begin{minipage}[t]{\linewidth}
\begin{itemize}[topsep=1pt,partopsep=.5ex,parsep=0ex, leftmargin=3mm]
\item basic operations run $\mathcal O(\log n)$ for average and worst-case times
\item can be linear in time, i.e., $\mathcal O(n)$, if tree is shape of linear chain
\end{itemize}
\vspace{.05cm}
\end{minipage}
}
\newcommand{\bstspace}{
\begin{minipage}[t]{\linewidth}
\begin{itemize}[topsep=1pt,partopsep=.5ex,parsep=0ex, leftmargin=3mm]
\item 
\end{itemize}
\vspace{.05cm}
\end{minipage}
}\begin{landscape}


\section*{Data Structure}
\begin{xltabular}{1.01\linewidth}{|l|X|X|X|X|}
\hline
Structure&Notes&Operations&Time Efficiency& Space Efficiency\\\hline
Stack (CLSR pg 232) & \stacknotes & \stackoperations & \stacktime & \stackspace \\\hline
Queue (CLSR pg 232) & \queuenotes & \queueoperations & \queuetime & \queuespace \\\hline
Binary Search Tree (CLSR pg 286)& \bstnotes & \bstoperations & \bsttime & \bstspace \\\hline
\end{xltabular}

%%% % code from http://rosettacode.org/wiki/Fibonacci_sequence#Python
%%% \begin{lstlisting}[label={list:first},caption=Sample Python code -- Fibonacci sequence calculated analytically.]
%%% from math import *
%%% 
%%% # define function 
%%% def analytic_fibonacci(n):
%%%   sqrt_5 = sqrt(5);
%%%   p = (1 + sqrt_5) / 2;
%%%   q = 1/p;
%%%   return int( (p**n + q**n) / sqrt_5 + 0.5 )
%%%  
%%% # define range
%%% for i in range(1,31):
%%%   print analytic_fibonacci(i)
%%% \end{lstlisting}
%%% 
%%% Following Listing~\ref{list:first}\ldots{} 
%%% Lorem ipsum dolor sit amet, consectetur adipiscing elit, sed do eiusmod tempor incididunt ut labore et dolore magna aliqua. Ut enim ad minim veniam, quis nostrud exercitation ullamco laboris nisi ut aliquip ex ea commodo consequat. Duis aute irure dolor in reprehenderit in voluptate velit esse cillum dolore eu fugiat nulla pariatur. Excepteur sint occaecat cupidatat non proident, sunt in culpa qui officia deserunt mollit anim id est laborum.
%%% 
%%% \section*{Problem 2}
%%% 
%%% \begin{lstlisting}[label={list:second},caption=Sample Bash code.]
%%% #! /bin/bash
%%% python stage1.py
%%% echo "Stage I done!"
%%% python stage2.py
%%% echo "Stage II done!"
%%% python stage3.py
%%% echo "Stage III done!"
%%% \end{lstlisting}
%%% 
%%% Lorem ipsum dolor sit amet, consectetur adipiscing elit, sed do eiusmod tempor incididunt ut labore et dolore magna aliqua. Ut enim ad minim veniam, quis nostrud exercitation ullamco laboris nisi ut aliquip ex ea commodo consequat. Duis aute irure dolor in reprehenderit in voluptate velit esse cillum dolore eu fugiat nulla pariatur. Excepteur sint occaecat cupidatat non proident, sunt in culpa qui officia deserunt mollit anim id est laborum.

\end{landscape}
\end{document}
